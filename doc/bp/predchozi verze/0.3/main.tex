\documentclass[FM,RP]{tulthesis}

\usepackage[FM,RP]{tulpack/tul}
\usepackage{fancyhdr}
\usepackage{graphicx} 
\usepackage{ifthen}
\usepackage{titlesec}
\usepackage{xcolor}
\usepackage{comment}

\TULtitle{Modování Minecraftu}{Minecraft modding}
\TULprogramme{B0613A140005}{Informační technologie}{Information Technology}
\TULbranch{}{Aplikovaná informatika}{Applied Informatics}
\TULauthor{Jiří Růta}
\TULsupervisor{Ing. Jana Vitvarová}
\TULyear{2024} \TULid{identifikační kód}


\begin{document}

 \ThesisStart{male}

\section{slovník}
blok(block) \\
vykopat(mine). Hr áč drží levé tlačítko myši, dokud mu blok nevypadne jako předmět. Předmět se poté přidá do jeho inventáře.\\
inventář (inventory) 
předmět (item)


\chapter{Minecraft}
V Minecraftu se celý svět skládá z bloků. Hráč tyto bloky může vykopat a poté je položit, anebo z nich vyrobit vybavení.
\par
Na začátku hry hráč vykope dřevo. Poté otevře UI svého inventář. Zde jsou všechny předměty, která má hráč u sebe. Kromě toho je zde UI ve kterém hráč může vyrábět předměty. Hráč si vyrobí dřevěnou sekeru, která mu umožní kopat dřeva rychleji.
\par
Hráč může vykopat každý blok ve hře. Vykopaný blok se mu poté přidá do inventáře. Doba kopání závisí na druhu bloku a na vybavení hráče.
\par
Hráčův inventář obsahuje všechny předměty, které má u sebe. Hráč si může otevřít UI svého inventáře kdykoli chce. Inventář v minecraftu je rozdělen na políčka. Každé políčko může obsahovat pouze jeden druh předmětu a pouze omezené množství.
\par
Součástí UI hráčova inventáře je i UI pro výrobu nových předmětů. Hráč v něm může vyrábět nové vybavení anebo bloky z věcí v jeho inventáři. Některé bloky mohou hráči povolit vyrobit si věci, které bez nich nemohl. Například pec umožní hráči předělat surovou rudu na ingoty.  

\chapter{Minecarft z programátorské stránky}
\section{Blok}
Minecraft server musí mít v paměti minimálně 2097152 bloků pro každého hráče připojeného na serveru. Každý blok je v paměti skladován jako odkaz na jeho objekt v registru bloků a odkaz na jeho blockstate. V registru bloků jsou uchovávané všechny neměnné vlastnosti a v blockstate uchovává všechny vlastnosti které se pro blok mohou měnit. 
\subsection{registr bloků}
Při startu hry se pro každý druh bloku zaregistruje objekt třídy \textit{Block} do registru. Tento objekt je neměnný. Veškeré instance daného bloku odkazují na tento jeden objekt. Jelikož je tento objekt společný pro všechny instance, pokud chceme ukládat data
specifická pro jeden block musíme použít blockstate anebo tile entity.
\subsection{blockstate}
Blockstate je neměnná třída obsahující pouze proměnné. Pro každou kombinaci proměnných je vygenerován singleton. Každý blok odkazuje na jeden ze singletonů. Blockstate šetří pamět, tím že dovoluje několika blokům držet pouze jednu promměnou (odkaz na blockstate) místo několika. \\Například máme blok lampy. Každý blok lampy odkazuje na stejný objekt v registru bloků, ale pokud chceme určit zdali je lampa zapnutá musíme se na který blockstate odkazuje.

\begin{figure}[h]
    \centering
    \includegraphics[width=0.5\linewidth]{tulpack/lamp.drawio.png}
    \caption{blockstate}
    \label{fig:blockstate}
\end{figure}

\subsection{tile entities}
Pokud v bloku potřebujeme skladovat velké množství proměných, můžeme využít \textit{tile entity}. Objekt třídy \textit{tile entity} je samostatný pro každý blok.\\
Například blok bedna umožnuje hráči odložit si předměty. Které předměty jsou uloženy v bedně je uloženo v \textit{tile entity}. Bedna také odkazuje na blockstate, který udává kterým směrem je otočena. A každá bedna také odkazuje na svůj objekt v registru bloků. 
\begin{figure}[h]
    \centering
    \includegraphics[width=0.5\linewidth]{chest.png}
    \caption{příklad vztahů}
    \label{fig:enter-label}
\end{figure}

\section{Inventář}
Inventář v minecraftu je rozdělen na políčka. Každé políčko může obsahovat pouze jeden druh předmětu a pouze omezené množství. V programu je každé políčko reprezentováno objektem \textit{itemstack}. 
\subsection{Itemstack}
Třída obsahuje 2 důležité proměnné, jaký předmět obsahuje (odkaz na jeho třídu) a jeho množství. Třída obsahuje převážně metody na porovnávání objektů Itemstack mezi sebou, předávání informocí o předmětu, ukládání a načítání dat.
\subsection{Item}
\begin{figure}[h]
    \centering
    \includegraphics[width=0.5\linewidth]{itemstack.drawio.png}
    \caption{ItemStack vztahy}
    \label{fig:enter-label}
\end{figure}
Každý druh předmětu má svůj singleton, který dědí třídu \textit{Item}. Tato třída obsahuje proměné a metody pro daný předmět. Například obsahuje metodu "onUse", která je volána pokud se hráč pokusí použít předmět. Jelikož je tato třída společná pro všechny itemy pokud chceme ukládat data specifická pro jeden předmět musíme použít NBT.
\begin{comment}
    Přidat pod item i výrobu
\end{comment}
\section{Výroba}
Velká část minecraftu je vytváření nových předmětů z těch které už hráč vlastní. Z čeho se předměty vyrábějí je zapsáno v JSON souborech. 
\subsection{Recept}
Ve hře je několik druhů receptů, každý druh receptu má svojí třídu implementující interface IRecipie. Každý JSON soubor ve složce s recepty má hodnotu "type", která říká kterou třídou ho má hra přečíst. Při zapnutí se načtou veškeré recepty z JSON souborů a pro každý json soubor se vytvoří jeden objekt třídy implementující IRecipie. Tyto objekty obsahují veškerá data z JSON souborů a implementují metody na porovnání zdali je možné recept vyrobit. Práce těchto tříd není vyrobit daný předmět, ale pouze říct zdali je možné daný předmět vyrobit a předat vyžádané informace.    
\section{Klient a Server}
Minecraft kód vždy běží dvakrát, jednou jako server a jednou jako klient. I když hráč hraje sám na vlastním počítači, minecraft pořád běží rozdělen na stranu serveru a klienta. Na obou stranách běží stejný kód. Na rozlišení se používá metoda "world.isRemote", která vrací \textit{true} pokud kód běží na klientu. Některé operace musí běžet pouze na jedné straně a druhou stranu upozornit pomocí paketů. Například vytváření vytváření nových předmětů by se musí provádět na serveru, jinak je možné že server nebude o předmětu vědět a když se ho hráč pokusí použít tak nebude fungovat, protože o něm server neví. Na druhou stranu, vykreslování grafiky se musí provádět pouze na klientu. 
\section{UI}
Některé bloky ,předměty otevřou hráči UI. Například když hráč klikne na pec, otevře se mu UI ve kterém může odložit předměty do pece, aby je pec předělala na jiné. Ui se v Minecarftu skládá ze dvou částí: Container a Screen.
\subsection{Container}
Container obstarává logicku a data. Komunikuje s ostatními částmi hry. Běží na serveru i na  clientu.
\subsection{Screen}
Určuje jak bude UI vypadat. Zobrazuje text a obrázky. Komunikuje pouze s Container. Může obsahovat logicku, ale měla by být jen okolo grafiky v UI. Běží pouze na klientu.

\chapter{Vytváření modu}
    Minecraft v základu nepovolujeme módování, pokud chceme musíme vytvořit mód musíme použít zavaděč módů, který se pustí při startu hry a načte kód módu do hry za pomoci \textit{Java Class Loader}. Zavaděč umožňuje přidat nové objekty do příslušných registrů, kde s nimi může komunikovat základní hra i veškeré módy.  Zavaděče módu také poskytují vývojářům módů události, které jsou volány během hry a vývojáři na ně mohou reagovat.Existují 2 zavaděče módu \textit{Forge} a \textit{Fabric}. Tyto zavaděče nemohou být spuštěny společně.
 \par   \textit{Forge} se zaměřuje na kompatibilitu mezi módy. \textit{Forge} má větší množství událostí a jsou k němu zabalené abstraktní třídy a rozhraní pro většinu věcí, které vývojáři módu často implementují. Díky tomu mají vývojáři normalizované rozhraní, která mohou používat ke komunikaci s ostatními módy. \textit{Forge} je také starší, takže většina módů je dělaná pro Forge.
\par    \textit{Fabric} se zaměřuje na výkon. \textit{Fabric} má menší API a na rozdíl od Forge načítá pouze kód který je potřeba. \textit{Fabric} také dovoluje upravovat soubory základní hry.
\par    Z pohledu hráče jsou zavaděče pouze omezení, které omezuje jaké módy hráč může využívat společně.
\par    Pro svůj mód jsem si vybral \textit{Forge}, jelikož se v něm jednodušeji řeší kompatibilita mezi ostatními módy a více módu je dělaných pro \textit{Forge}.
\par    Minecraft vyšel v roce 2009, od té doby vyšlo 20 velkých aktualizací. Minecraft dovoluje hráčům hrát jakoukoli verzi. Jelikož se s každou verzí minecraftu mění kód základní hry, musí pro každou verzi udělat nová verze zavaděče módu. Módy napsané pro jednu verzi minecraftu nejsou kompatibilní s ostatními verzemi. Proto se při verzování  módu udává hned za verzí módu verze minecraftu pro kterou je napsán. Například mód\textit{ Ex Aqua 0.4 - 1.16.5}  je mód pro verzi minecraft 1.16.5 a verze samotného módu je 0.4 . Některé módy jsou dělané na více verzí Minecraftu a jejich autor je pravidelně předělává na nové verze. Když je mód hodně populární, ale autor ho nechce předělat na ostatní verze Minecraftu, často se najde jiný programátor který udělá podobný mód na jiné verze, proto pro staré populární módy je často několik neoficiálních pokračování.   
\par    Před vytvářením módu je potřeba se rozhodnout pro jakou verzi minecraftu a pro jaký zavaděč bude programátor mód vyvíjet. Oba zavaděče nabízejí šablonu na stažení. Šablona obsahuje \textit{Gradle} soubory, které stáhnou všechny soubory potřebné pro zavaděč a také celý zdrojový kód vybrané verze Minecraftu. 
\par    Minecraft je produkt společnosti Microsoft. Microsoft dovoluje používat zdrojové kódy jejich hry pro vývoj módů ,dokud vývojář šíří pouze svůj mód a nešíří zdrojové kódy hry Minecraft. Vývojář je vlastníkem svého kódu, ale nesmí mód prodávat. Může ,ale žádat o peněžní dary. Oba zavaděče nijak právně neomezují  vývojáře.
\par    Po přípravě šablony si vývojář musí zvolit id pro svůj mód. Id je textový řetězec, který nesmí obsahovat speciální znaky kromě „\textit{\_}“.  Tento textový řetězec bude použit jako část cesty k souborům módu a také pro identifikaci módu. 
\par    Všechny módy dodržují stejnou hierarchii souborů. Mód je rozdělen na složku \textit{java} a složku \textit{resources}.  Pod složkou \textit{java} se nachází složka se jménem autora a pod ní složka pojmenovaná jako id módu, v ní se nachází veškerý Java kód. Ve složce \textit{resources} jsou ostatní soubory rozděleny na \textit{assets} a \textit{data}. Složka \textit{assets} obsahuje soubory, které by měli být potřebné pouze na klientu, zatímco složka \textit{data} obsahuje soubory potřebné na klientu a serveru.
\par    Ve složce \textit{resources/assets/blocksates} se nachází Json soubory, které řikají jaký model má být pro blok použit na základě jeho \textit{blockstate} proměných. Pro každý blok je jeden tento soubor. Jméno tohoto souboru odpovídá jménu pod kterým je blok registrován v registru bloků. Ve složce  \textit{resources/assets/models} jsou Json soubory  obsahují modely pro bloky a předměty. Soubor také obsahuje odkazy na textury použité v modelu. Textury se nacházejí ve složce \textit{resources/assets/textures}. Ve složce \textit{resources/assets/lang} je Json soubor pro každý jazyk, obsahující klíč a jeho překlad. Ve  složce \textit{resources/assets/gui} je uložena grafika využitá v UI.
\par 

\begin{comment}
\textit{Named binary tag} je jednoduchý hiearchický formát pro ukládání dat.
\end{comment}

\chapter{Můj mód z pohledu hráče}
\par  Módy často přidávají bloky které představují různé stroje, jako třeba elektrická pec anebo Hydraulický lis. Tyto bloky většinou mají vlastní inventář do kterého dá hráč předměty a stroj z nich vyrobí jiné. Tyto stroje většinou potřebují elektřinu, aby fungovali, což je věc která se v základní hře nevyskytuje a je exkluzivní pouze pro módy.
 \par   Těmto módům se říká technické módy. Dobrý technický mód funguje s ostatními technickými módy. Například pokud mód přidává generátor tak mělo být možno z generátoru odvádět elektřinu i kabely z cizího módu a napájet s ní stroje z cizích módů. 
 \par   Můj mód přidává několik strojů. Přidává síta, lis a kotlík. Síto přesívá tekutiny na rudy. Lis umí předměty na tekutiny. V kotlíku může hráč kombinovat tekutiny a předměty, aby získal nové tekutiny.  Všechny fungují s trubkami a kabely z populárních technických módů.
\par    Můj mód je inspirován módem\textit{ Ex Nihilo}. Jedná se o populárním mód vydaný v roce 2014. Tento mód přidává síta ve která přesívají hlínu na rudy a sud ve kterém může hráč kombinovat tekutiny a předměty, aby získal nové tekutiny. Tyto bloky nepotřebují elektřinu.
 \par   Tento mód je neudržovaný a funguje pouze pro velmi starou verzi minecraftu, ale má několik neoficiálních pokračování. 
 \par       Ex Nihilo 2 je jediné oficiální pokračování k módu \textit{ Ex Nihilo} od jeho autora. Bylo vydáno roku 2015, ale od roku 2016 je opuštěné. Obsahově je podobné původnímu \textit{Ex Nihilo}, ale některé věci chybí, mód zůstal nedodělán.
 \par       Ex Nihilo Omnia je neoficiální pokračování vydané 4.9. 2016. Tento mód přidává vše co původní \textit{ Ex Nihilo} pouze s malými úpravami a obsahem navíc. Tento mód dostal svoji poslední aktualizaci v roce 2019.
Ex Nihilo Adscensio  je neoficiální pokračování vydané 29.11 2016. Tento mód je první který přidal více úrovní sít. Poslední aktualizaci dostal 6.5 2017.
\par    \textit{Ex Nihilo Creatio} je fork \textit{Ex Nihilo Adscensio}. Je to první mód který přidává sívky, které hráč musí napájet elektřinou, ale za odměnu vyrábějí bez hráčovi přítomnosti. Mód byl vydán 14.8. 2017 a poslední aktualizace byla vydána 27.8 2019.
\par    Ex Nihilo: Sequentia začalo jako  \textit{Ex Nihilo Creatio} předělané na vyšší minecraft verze. Tento mód obsahuje veškerý obsah ze všech předchozích Ex Nihilo inspirovaných módů. Byl vydán 4.8. 2020 a je stále podporován. Má několik módů které dodávají další obsah k tomuto módu anebo přidávají kompatibilitu s ostatními módy. 
\par    Fabricae Ex Nihilo neoficiální pokračování módu  \textit{Ex Nihilo} pro zavaděč Fabric. Vydáno 21.3. 2022 a stále aktualizováno.

\begin{table}[]
\begin{tabular}{lllr}
\hline
\multicolumn{1}{|l|}{jméno módu} & \multicolumn{1}{l|}{podporované minecraft verze} & \multicolumn{1}{l|}{zavaděč} & \multicolumn{1}{l|}{počet stažení} \\ \hline
Ex Nihilo                        & 1.6 a 1.7                                        & Forge                        & 8 058 815                          \\
Ex Nihilo 2                      & 1.8                                              & Forge                        & 61 152                             \\
Ex Nihilo Omnia                  & 1.10 a 1.12                                      & Forge                        & 2 611 291                          \\
Ex Nihilo Adscensio              & 1.10                                             & Forge                        & 6 236 591                          \\
Ex Nihilo: Creatio               & 1.12                                             & Forge                        & 15 638 553                         \\
Ex Nihilo: Sequentia             & 1.15 až 1.20 (nejnovější)                        & Forge                        & 6 761 281                          \\
Fabricae Ex Nihilo               & 1.18 až 1.20 (nejnovější)                        & Fabric                       & 168 881                            \\
Ex Aqua                          & 1.16                                             & Forge                        & 432                               
\end{tabular}
\end{table}
        
        

\begin{comment}
    Můj mod je inspirován \textit{Ex Nihilo} módem. \textit{Ex Nihilo} je populární mód, který umožnuje získávat hráčům předměty z ničeho, skrz automatizaci a recyklaci. \textit{Ex Nihilo} je často používán společně s jinými módy v žánru \textit{skyblock}. V tomto žánru se hráč nachází na malém ostrově tvořeného z malého množství základních bloků. Jeho cílem je využít jeho omezené zdroje k získání nových zdrojů a rozšíření ostrovu. Většinou ostrov obsahuje hlínu a strom, který hráč může pokácet a poté znovu zasadit, čímž získá teoreticky nekonečné množství dřeva a listí. Ex Nihilo přidává do hry kompostér a sívky. Kompostér umožnuje hráči předělat listí na hlínu, Sívky umožnují hráči přesívat hlínu na rudy.
\end{comment}


\chapter{Můj mód z pohledu programátora}
\par    Začal jsem tím že jsem do hry přidal blok pro sívku. Nejprve jsem vytvořil třídu pro blok která dědí ze třídy  \textit{Block}. Poté jsem  registroval objekt s touto třídou do registru bloků. Registrace vyžaduje jméno pro tento objekt a  construktor třídy \textit{Block}. Podle jména hra pak vyhledává stejnojmené soubory v podsložkách \textit{resources}.  Pro blok je potřeba do přidat soubor do složek : \textit{blockstates}, \textit{models}, \textit{textures}. Do složky \textit{blockstates} přidám json soubor který obsahuje, který model má být použit v závislosti na proměnných v \textit{blockstate}.  Do složky model přidám json soubor, který obsahuje model bloku a odkaz na jeho textury. Soubor také může dědit z ostatních souborů, pomocí parametru \textit{parent}. Pro model krychle je zde předpřipravený soubor \textit{cube_all}. Ve složce \textit{textures} se nacházejí všechny textury.
\begin{comment}
    Tímto jsem přidal do hry blok, ten ale nemá žádné funkce.
\end{comment}

 \par   Pro upravení modelu bloku je potřeba přepsat metodu \textit{getShape} ve třídě bloku ,aby vracela model sestavený z kvádrů a také upravit soubor ve složce \textit{models}. K vytvoření modelů pro moje blocky jsem použil program \textit{Blockbench}.

\par    Dále jsem k bloku přidal \textit{tile entity}. Nejprve jsem vytvořil třídu dědící od třídy \textit{TileEntity}. V této třídě jsem přepsal metody k ukládání a načítání dat z nbt. Poté jsem \textit{tile entity} registroval. Při registraci se předá registru constructor pro tileentity a odkaz na všechny registrované bloky kterým tileentity patří. Poté přepíšu ve třídách těch bloků metodu na vrácení \textit{tile entity} a propojení bloku a tile entity je kompletní. 
\par    Chtěl jsem v tile entity skladovat tekutity tak jsem si vytvořil novou třídu  \textit{MyLiquidTank} která dědila ze třídy \textit{FluidTank}. Implementoval jsem všechny její metody a přidal pár vlastních abych zabránil duplikacy kódu. Objekt této třídy jsem poté přidal do své\textit{ tile entity}. Pro práci s ním jsem vytvořil interface \textit{IMyLiquidTankTile} . 
\par    Jelikož potřebuju synchronizovat \textit{MyLiquidTank} mezi server a clientem vytvořil jsem si třídu  \textit{MyFluidStackPacket} která je poslána ze servru klientovy a obsahuje momentální stav daného \textit{MyLiquidTank}. Ve třídě MyLiquidTank jsem implementoval odeslání a přijmutí tohoto packetu. 
 \par   Také jsem chtěl aby ostatní módy mohli komunikovat s \textit{MyLiquidTank} tak jsem vytvořil třídu \textit{WaterFluidTankCapabilityAdapter} co slouží jako adaptér. Která překládá můj  \textit{MyLiquidTank} na\textit{ IFluidTank} což je interface zabudovaný v Forge. Testoval jsem tento kód s nejznámějšími módy a se všemi fungoval.
 \par   Poté jsem to samé udělal pro \textit{MyEnergyStorage}. Vytvořil jsem  \textit{IMYEnergyStorageTile} pro \textit{tile entity} obsahující  \textit{MyEnergyStorage}.  Vytvořil jsem \textit{MyEnergyPacket} pro synchronizaci klienta a servru. Vytvořil jsem \textit{EnergyStorageAdapter} který překládá \textit{MyEnergyStorage} na IEnergyStorage z Forge pro komunikaci s ostatními módy.
\par    Poté jsem měl\textit{ tile entity} která je schopná skladovat tekutiny, elekřinu a předměty \begin{comment}
        předměty jsem nikdy nezmínil
\end{comment}
 \par   Dále jsem chtěl aby v\textit{ tile entity} bylo možné vyrábět předměty, k tomu jsem potřeboval recept který bude uvádět co bude potřeba jako vstup do stroje a co vyjde jako výstup. Vytvořil jsem třídu \textit{SieveRecipe} implementující interface \textit{ISieveRecipe}. Tato třída přečte data z json souboru a poté vytvoří jednu svoji instanci pro každý úspěšně načtený soubor. Jedna instance této třídy představuje jeden recept. Tato třída obsahuje metody které určí zdali jsou všechny potřebné ingredience a vrátí předmět, který má být vyroben.
\par    Ke každému druhu receptu jsem musel napsat třídu která překládá daný recept pro jei mód a přidat mu UI.
\par    Také jsem pro \textit{tile entity } přidal container a screen. container bere informace přímo z tile entity a předáváje pro screen. container také obsahuje logicku.
    

\begin{comment}
Při programování sívky jsem začal vytvořením třídy bloku sívky a jeho registrací do registrů. Tato třída odkazuje na \textit{tile entity} a UI kontejner. Dále jsem vytvořil vlastní implementace \textit{forge} rozhraní \textit{IEnergyStorage} a \textit{IFluidTank}. Tyto třídy slouží k ukládání elektřiny a vody. 
\end{comment}
\end{document}
